\section{Introduction}
The \textbf{heat equation} in one dimension (1D) is a partial differential equation (PDE) that describes how heat (or temperature) distributes over time in a given medium. It is given by:

\begin{equation}
\frac{\partial u}{\partial t} = \alpha \frac{\partial^2 u}{\partial x^2}
\end{equation}

where:
\begin{itemize}
    \item $u(x,t)$ represents the temperature at position $x$ and time $t$.
    \item $\alpha$ is the \textbf{thermal diffusivity} of the material, given by $\alpha = \frac{k}{\rho c}$, where $k$ is the thermal conductivity, $\rho$ is the density, and $c$ is the specific heat capacity.
    \item $\frac{\partial u}{\partial t}$ represents the \textbf{rate of change of temperature} with respect to time.
    \item $\frac{\partial^2 u}{\partial x^2}$ represents the \textbf{spatial curvature of the temperature distribution}, which determines how heat diffuses.
\end{itemize}

\section{Boundary and Initial Conditions}
To solve the heat equation, we need boundary and initial conditions.

\subsection{Initial Condition}
Specifies the temperature distribution at $t = 0$:
\begin{equation}
    u(x, 0) = f(x)
\end{equation}
where $f(x)$ is a given function.

\subsection{Boundary Conditions}
Define how heat behaves at the boundaries. Common types include:

\begin{itemize}
    \item \textbf{Dirichlet condition}: Fixed temperature at boundaries,
    \begin{equation}
        u(0, t) = T_1, \quad u(L, t) = T_2.
    \end{equation}
    
    \item \textbf{Neumann condition}: Specifies the heat flux (insulated or heat flow),
    \begin{equation}
        \frac{\partial u}{\partial x} \bigg|_{x=0} = q_1, \quad \frac{\partial u}{\partial x} \bigg|_{x=L} = q_2.
    \end{equation}
    
    \item \textbf{Mixed condition}: A combination of Dirichlet and Neumann conditions.
\end{itemize}

\section{Solution Methods}
There are several methods to solve the heat equation:
\begin{itemize}
    \item \textbf{Separation of Variables}: Used for simple cases with homogeneous boundary conditions.
    \item \textbf{Fourier Series}: Expands the solution in sine or cosine series.
    \item \textbf{Fourier Transform}: Used for problems on infinite domains.
    \item \textbf{Numerical Methods}: Finite difference, finite element, and finite volume methods for complex geometries and varying properties.
\end{itemize}



%%%%%%%%%%%%%%%%%%%%%%%%%%%%%%%%%%%%%%%%%%%%%%%%%%%
%%%%%%%%%%%%%%%%%%%%%%%%%%%%%%%%%%%%%%%%%%%%%%%%%%%

\section{Numerical Algorithms}
There are several numerical algorithms to solve the 1D heat equation.

\subsection{Finite Difference Method (FDM)}
The finite difference method discretizes the heat equation using numerical approximations for derivatives.

\subsubsection{Explicit Method (FTCS)}
Using a forward difference for time and a centered difference for space, the heat equation:
\begin{equation}
\frac{\partial u}{\partial t} = \alpha \frac{\partial^2 u}{\partial x^2}
\end{equation}

is discretized as:
\begin{equation}
    u_i^{n+1} = u_i^n + \lambda (u_{i+1}^n - 2u_i^n + u_{i-1}^n)
\end{equation}

where $\lambda = (\alpha \Delta t)/\Delta x^2$ is the stability parameter. The explicit method is conditionally stable if:
\begin{equation}
    \lambda \leq \frac{1}{2}
\end{equation}

\subsubsection{Implicit Method (BTCS)}
Using a backward difference for time:
\begin{equation}
    u_i^{n+1} - \lambda (u_{i+1}^{n+1} - 2u_i^{n+1} + u_{i-1}^{n+1}) = u_i^n
\end{equation}
This leads to a system of linear equations that can be solved using matrix inversion. The implicit method is unconditionally stable.

\subsubsection{Crank-Nicolson Method (CN)}
A combination of explicit and implicit methods:
\begin{equation}
    u_i^{n+1} - \frac{\lambda}{2} (u_{i+1}^{n+1} - 2u_i^{n+1} + u_{i-1}^{n+1}) = u_i^n + \frac{\lambda}{2} (u_{i+1}^n - 2u_i^n + u_{i-1}^n)
\end{equation}
This method is unconditionally stable and second-order accurate.

\subsection{Finite Element Method (FEM)}
The finite element method divides the domain into elements and approximates $u(x,t)$ using basis functions.

\subsection{Finite Volume Method (FVM)}
The finite volume method applies the integral form of the heat equation to control volumes and approximates heat fluxes across volume boundaries.

\section{Comparison of Methods}

\begin{table}[h]
    \centering
    \begin{tabular}{|c|c|c|c|c|}
        \hline
        Method & Stability & Accuracy & Complexity & Application \\
        \hline
        Explicit (FTCS) & Conditional ($\lambda \leq 1/2$) & First-order & Simple & Fast, but unstable for large time steps \\
        Implicit (BTCS) & Unconditional & First-order & Medium & More stable, requires solving a system \\
        Crank-Nicolson & Unconditional & Second-order & High & Best accuracy, but computationally expensive \\
        Finite Element & Unconditional & Second-order+ & High & Best for complex geometries \\
        Finite Volume & Unconditional & Second-order+ & High & Best for conservation laws \\
        \hline
    \end{tabular}
    \caption{Comparison of Numerical Methods}
\end{table}

